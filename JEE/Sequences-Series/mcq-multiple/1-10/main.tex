\iffalse
\title{Sequence and Series}
\author{DHAWAL-ee24btech11015}
\section{mcq-multiple}
\fi

	\item If the first and the $\brak{2n-1}^{th}$ terms of an A.P., a G.P. and an H.P. are equal and their $n^{th}$ terms are $a, b \; \text{and} \; c$ respectively, then \hfill\brak{1988-2 Marks}

\begin{multicols}{4}
\begin{enumerate}
\item $a=b=c$
\item $a \geq b \geq c$
\item $a+b=c$
\item $ac-b^2=0$
\end{enumerate}
\end{multicols}

\item For $0 < \phi < \pi /2$, if 
\begin{align*}
x=\sum_{n=0}^{\infty} \cos^{2n} \phi , \;
y=\sum_{n=0}^{\infty} \sin^{2n} \phi , \;
z=\sum_{n=0}^{\infty} \cos^{2n} \phi \sin^{2n} \phi
\end{align*}
then: \hfill\brak{1993-2 Marks}
\begin{multicols}{4}
\begin{enumerate}
\item $xyz=xz+y$
\item $xyz=xy+z$
\item $xyz=x+y+z$
\item $xyz+yz+x$
\end{enumerate}
\end{multicols}

\item Let $n$ be a odd integer. If 
\begin{align*}
\sin n\theta= \sum_{r=0}^{n} b_r \sin^{r} \theta, 
\end{align*}
for every value of $\theta$, then
\hfill\brak{1998-2 Marks}
\begin{multicols}{2}
\begin{enumerate}
\item $b_0=1, b_1=3$
\item $b_0=0, b_1=n$
\item $b_0=-1, b_1=3$
\item $b_0=0, b_1=n^2-3n+3$
\end{enumerate}
\end{multicols}

\item Let $T_r$ be the $r^{th}$ term of an A.P., for $r=1,2,3,\dots$ If for some positive integers $m,n$ we have
$T_m=\frac{1}{n}$ and $T_n=\frac{1}{m}$ ,then $T_{mn}$ equals \hfill\brak{1998-2 Marks}

\begin{multicols}{4}
\begin{enumerate}
\item $\frac{1}{mn}$
\item $\frac{1}{m} + \frac{1}{m}$
\item $1$
\item $0$
\end{enumerate}
\end{multicols}

\item If $x>1,y>1,z>1$ are in G.P.,then $\frac{1}{1+\ln x},\frac{1}{1+\ln y},\frac{1}{1+\ln z}$ are in 
\hfill\brak{1998-2 Marks}
\begin{multicols}{4}
\begin{enumerate}
\item A.P.
\item H.P.
\item G.P.
\item None of these
\end{enumerate}
\end{multicols}

\item For a positive integer $n$, let
$a_n=1+\frac{1}{2}+\frac{1}{3}+\frac{1}{4}+\dots\frac{1}{(2^n)-1}$. Then \hfill\brak{1999-2 Marks}
\begin{multicols}{4}
\begin{enumerate}
\item $a100)\leq 100$
\item $a(100) > 100$
\item $a(200)\leq 100$
\item $a(200) > 100$
\end{enumerate}
\end{multicols}

\item A straight line through the vertex $\vec{P}$ of a triangle $\Delta PQR$ intersects the side QR at the point $\vec{S}$ and the circumcircle of the triangle $\Delta{PQR}$ at the point $\vec{T}$. If $\vec{S}$ is not the centre of the circumcircle,then  \hfill\brak{2008}
\begin{multicols}{2}
\begin{enumerate}
\item $\frac{1}{PS}+\frac{1}{ST}<\frac{2}{\sqrt{QS \cdot SR}}$
\item $\frac{1}{PS}+\frac{1}{ST}>\frac{2}{\sqrt{QS \cdot SR}}$
\item $\frac{1}{PS}+\frac{1}{ST}<\frac{4}{QR}$
\item $\frac{1}{PS}+\frac{1}{ST}>\frac{4}{QR}$
\end{enumerate}
\end{multicols}

\item Let 
\begin{align*}
S_n=\sum_{k=1}^{n}\frac{n}{n^2+kn+k^2} \text{ and }   T_n=\sum_{k=0}^{n-1}\frac{n}{n^2+kn+k^2}
\end{align*}
for $n=1,2,3,\dots$ Then,\hfill\brak{2008}
\begin{multicols}{4}
\begin{enumerate}
\item $S_n<\frac{\pi}{3\sqrt{3}}$
\item $S_n>\frac{\pi}{3\sqrt{3}}$
\item $T_n<\frac{\pi}{3\sqrt{3}}$
\item $T_n>\frac{\pi}{3\sqrt{3}}$
\end{enumerate}
\end{multicols}

\item Let \begin{align*} S_n=\sum_{k=1}^{4n}\brak{-1}^\frac{k\brak{k+1}}{2}k^2.\end{align*}  Then $S_n$ can take value(s)  \hfill\brak{JEE Adv.2013}
\begin{multicols}{4}
\begin{enumerate}
\item $1056$
\item $1088$
\item $1120$
\item $1332$
\end{enumerate}
\end{multicols}

\item Let $\alpha$ and $\beta$ be the roots of $x^2-x-1=0$, with $\alpha>\beta$. For all positive integers $n$, define
\begin{align*}
a_n=\frac{\alpha_n-\beta_n}{\alpha-\beta},n\geq2
b_1=1  \text{ and }  b_n=a_{n-1}+a_{n+1},n\geq1
\end{align*}
Then which of the following options is/are correct?
\hfill\brak{JEE Adv. 2019}
\begin{multicols}{2}
\begin{enumerate}
\item $\sum_{n=1}^{\infty}\frac{a_n}{10^n}=\frac{10}{89}$
\item $B_n=a^n+b^n \; \forall /; n\geq1$
\item $a_1+a_2+a_3+.....a_n=a_{n+2}-1 \forall n\geq1$
\item $\sum_{n=1}^{\infty}\frac{b_n}{10^n}=\frac{8}{89}$
\end{enumerate}
\end{multicols}




