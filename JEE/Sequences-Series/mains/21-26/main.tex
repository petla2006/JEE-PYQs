\iffalse
\title{Assignment1}
\author{Dasari Manikanta}
\section{mains}
\fi

\item The number of integers greater than $6,000$ that can be formed, using digits $3$,$5$,$6$,$7$ and $8$,without repetition,is\hfill{[JEE M 2015]}
    \begin{enumerate}
    \item $120$ 
    \item $72$
    \item $216$
    \item $192$ 
    \end{enumerate} 
    
	    \item If all words (with or without) having five letters,formed using the letters of the word SMALL and arranged as in a dictionary;then the position of the word SMALL is; \hfill{[JEE M 2015]}
\begin{enumerate}
    \item $ 52^{nd} $
    \item $ 58^{th} $
    \item $ 46^{th} $
    \item $ 59^{th} $
    \end{enumerate} 
 
	 \item A man $X$ has $7$ friends, $4$ of them are ladies and $3$ are men. His wife $Y$ also has $7$ friends, $3$ of them are ladies and $4$ are men. Assume $X$ and $Y$ have no common friends.Then the total number of ways in which $X$ and $Y$ together can throw a party inviting $3$ ladies and $3$ men, so that $3$ friends of each of $X$ and $Y$ are in this part$y$, is: \hfill{[JEE M 2017]}
\begin{enumerate}
\item $484$ 
\item $485$
\item $468$
\item $469$
\end{enumerate}

	\item From $6$ different novels and $3$ different dictionaries,$4$ novels and $1$ dictionary are to be selected and arranged in a row on a shelf so that  the dictionary is always in the middle. The number of such arrangements is:\hfill{[JEE M 2018]}
 \begin{enumerate}
     \item less than $500$ 
     \item at least $500$ but less than $750$
     \item at least $750$ but less than $1000$
     \item at least $1000$
     \end{enumerate}

	\item Consider a class of $5$ girls and $7$ boys. The number of different teams consisting of $2$ girls and $3$ boys that can be formed from this class,if there are two specific boys $A$ and $B$,who refuse to be members of the same team,is:\hfill{[JEE M 2019-9 Jan(M)]}
\begin{enumerate}
    \item $500$  
    \item $200$
    \item $300$
    \item $350$
    \end{enumerate}

	\item A committee of $11$ members is to be formed from $8$ males and $5$ females. If m is the number of ways the committee is formed with at least $6$ males and n is the number of ways the committee is formed with at least $3$ females, is:
		\hfill{[JEE M 2019-9April(M)]}
\begin{enumerate}
      \item $m+n=68$ 
      \item $m=n=78$
      \item $n=m-8$
      \item $m=n=68$
  \end{enumerate}  
