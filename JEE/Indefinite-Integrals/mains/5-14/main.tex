\iffalse
  \title{INDEFINITE INTEGRALS}
  \author{BADDE VIJAYA SREYAS}
  \section{mains}
\fi

% \begin{enumerate}
%5
	\item The value of $\sqrt{2}\int\frac{\sin xdx}{\sin \brak{x-\frac{\pi}{4}}}$

		\hfill{(2008)}

		(a)$x+\log\abs{\cos \brak{x-\frac{\pi}{4}}}+c$

		(b)$x-\log\abs{\sin \brak{x-\frac{\pi}{4}}}+c$

		(c)$x+\log\abs{\sin \brak{x-\frac{\pi}{4}}}+c$

		(d)$x-\log\abs{\cos \brak{x-\frac{\pi}{4}}}+c$

%6
	\item If the $\int\frac{5\tan x}{\tan x-2}dx=x+a\ln \abs{\sin x-2\cos x}+k$, then $a$ is equal to

		\hfill{(2018)}

		\begin{multicols}{4}
			\begin{enumerate}
				\item -1
				\item 2
				\item 1
				\item 2
			\end{enumerate}
		\end{multicols}
		
%7
	\item If $\int f\brak{x}dx=\psi\brak{x}$, then $\int x^5f\brak{x^3}dx$ is equal to:

		\hfill{(JEE M 2013)}

		(a) $\frac{1}{3}\sbrak{x^3\psi\brak{x^3}-\int x^2\psi(x^3)dx} + C$

		(b) $\frac{1}{3}x^3\psi\brak{x^3}-3\int x^3\psi(x^3)dx + C$

		(c) $\frac{1}{3}x^3\psi\brak{x^3}-\int x^2\psi(x^3)dx + C$

		(d) $\frac{1}{3}\sbrak{x^3\psi\brak{x^3}-\int x^3\psi(x^3)dx} + C$

%8
	\item The integral $\int\brak{1+x-\frac{1}{x}}e^{x+\frac{1}{x}}dx$ is equal to

		\hfill{(JEE M 2014)}

		\begin{multicols}{2}
			\begin{enumerate}
				\item $\brak{x+1}e^{x+\frac{1}{x}}+c$
				\item $-xe^{x+\frac{1}{x}}+c$
				\item $\brak{x-1}e^{x+\frac{1}{x}}+c$
				\item $xe^{x+\frac{1}{x}}+c$
			\end{enumerate}
		\end{multicols}

%9
	\item The integral $\int\frac{dx}{x^2(x^4+1)^{3/4}}$ equals:

		\hfill{(JEE M 2015)}

		\begin{multicols}{2}
			\begin{enumerate}
				\item $-\brak{x^4+1}^{\frac{1}{4}}+c$
				\item $-\brak{\frac{x^4+1}{x^4}}+c$
				\item $\brak{\frac{x^4+1}{x^4}}^{\frac{1}{4}}+c$
				\item $\brak{x^4+1}^{\frac{1}{4}}+c$
			\end{enumerate}
		\end{multicols}

%10
	\item The integral $\int\frac{2x^{12}+5x^9}{\brak{x^5+x^3+1}^3}dx$ is equal to

		\hfill{(JEE M 2016)}

		\begin{multicols}{2}
			\begin{enumerate}
				\item $\frac{x^5}{2\brak{x^5-x^3+1}^2}+C$
				\item $\frac{-x^{10}}{2\brak{x^5+x^3+1}^2}+C$
				\item $\frac{-x^5}{\brak{x^5+x^3+1}^2}+C$
				\item $\frac{x^{10}}{2\brak{x^5+x^3+1}}+C$
			\end{enumerate}
		\end{multicols}
		where $C$ is an arbitrary constant

%11
	\item Let I$_n$ = $\int$tan$^x$dx, \brak{n>1}. I$_4$+I$_6$ = $a \tan^5x + bx^5 + C$, where $C$ is constant of integration, then the ordered pair \brak{a, b} is equal to :

		\hfill{(JEE M 2017)}

		\begin{multicols}{4}
			\begin{enumerate}
				\item$\brak{-\frac{1}{5}, 0}$
					
				\item$\brak{-\frac{1}{5}, 1}$
					
				\item$\brak{\frac{1}{5}, 0}$
					
				\item$\brak{\frac{1}{5}, -1}$
			\end{enumerate}
		\end{multicols}
		
%12
	\item The integral $\int\frac{\sin^2x\cos^2x}{\brak{\sin^5x+\cos^3x\sin^2x+\sin^3x\cos^2x+\cos^5x}^2}$dx is equal to

		\hfill{(JEE M 2018)}
		
		\begin{multicols}{2}
			\begin{enumerate}
				\item $\frac{-1}{3\brak{1+\tan ^3x}}$ + $C$
				\item $\frac{1}{1+\cot ^3x}$ + $C$
				\item $\frac{-1}{1+\cot ^3x}$ + $C$
				\item $\frac{1}{3\brak{1+\tan ^3x}}$ + $C$
			\end{enumerate}
		\end{multicols}
		
%13
	\item For $x^2\neq$ n$\pi+1$, n$\in$N (the set of natural numbers), the integral $\int x\sqrt{\frac{2\sin \brak{x^2-1}-\sin 2\brak{x^2-1}}{2\sin \brak{x^2-1}+\sin 2\brak{x^2-1}}}$dx is equal to:

		\hfill{(JEE M 2019 - 9 Jan(M))}

		\begin{multicols}{1}
			\begin{enumerate}
				\item $\log_e\abs{\frac{1}{2}\sec ^2\brak{x^2-1}} + c$
				\item $\frac{1}{2}\log_e\abs{\sec ^2\brak{\frac{x^2-1}{2}}} + c$
				\item $\frac{1}{2}\log_e\abs{\sec ^2\brak{\frac{x^2-1}{2}}} + c$
				\item $\log_2\abs{\sec \brak{\frac{x^2-1}{2}}} + c$
			\end{enumerate}
		\end{multicols}
		(where $c$ is a constant of integration)

%14
	\item The integral $\int \sec ^{2/3}x\cosec ^{4/3}x dx$ is equal to

		\hfill{(JEE M 2019 - 9 April (M))}

		\begin{multicols}{2}
			\begin{enumerate}
				\item -3$\tan ^{-1/3}x+C$
				\item -$\frac{3}{4}\tan ^{-4/3}x+C$
				\item -3$\cot ^{-1/3}x+C$
				\item 3$\tan ^{-1/3}+C$
			\end{enumerate}
		\end{multicols}
		(Here, $C$ is a constant of integration)

% \end{enumerate}
