\iffalse
\title{5B(1-7)
\author{EE24BTECH11055 - Sai Akhila Reddy Turpu}
\section{mcq-single}
\fi
%\begin{enumerate}
	\item The coefficients of $x^p$ and $x^q$ in the expansion of $\brak{1+x}^{p+q}$ are:
	\hfill{(2002)}
	\begin{enumerate}
		\item equal
		\item equal with opposite signs
		\item reciprocals of each other
		\item none of these
	\end{enumerate}
\item If the sum of coefficients in the expansion of $\brak{a+b}^n$ is 4096, then the greatest coefficient in the expansion is:  
	\hfill{\brak{2002}}
	\begin{enumerate}
			\begin{multicols}{4}
		\item $1594$
		\item $792$
		\item $924$
		\item $2924$
			\end{multicols}
	\end{enumerate}
\item The positive integer just greater than \\
$(1+0.0001)^{10000}$ is: 
		\hfill{\brak{2002}}
	\begin{enumerate}
			\begin{multicols}{4}
		\item $4$
		\item $5$
		\item $2$
		\item $3$
			\end{multicols}
	\end{enumerate}
\item $r$ and $n$ are positive integers, $r>1, n>2$ and coefficient of $\brak{r+2}^{th}$ term and $\brak{3r}^{th}$ term in the expansion of $\brak{1+x}^{2n}$ are equal, then n equals:

	\hfill{\brak{2002}}
	\begin{enumerate}
			\begin{multicols}{4}
		\item $3r$
		\item $3r+1$
		\item $2r$
		\item $2r+1$
			\end{multicols}
	\end{enumerate}
\item If $a_n = \sqrt{7+\sqrt{7+\sqrt{7+...}}}$ having $n$ radical signs, then by methods of mathematical induction, which is true?
	\hfill{\brak{2002}}
	\begin{enumerate}
			\begin{multicols}{2}
		\item $a_n > 7$  $\forall$ $n \ge 1$
		\item $a_n < 7$  $\forall$ $n \ge 1$
		\item $a_n < 4$  $\forall$ $n \ge 1$
		\item $a_n < 3$  $\forall$ $n \ge 1$
			\end{multicols}
	\end{enumerate}
\item If $x$ is positive, the first negative term in the expansion of $\brak{1+x}^\frac{27}{5}$ is:
	\hfill{\brak{2003}}
	\begin{enumerate}
			\begin{multicols}{4}
		\item$6th$ $term$
		\item$7th$ $term$
		\item$5th$ $term$
		\item$8th$ $term$
			\end{multicols}
	\end{enumerate}
\item The number of integral terms in the expansion of $\brak{\sqrt{3}+\sqrt[8]{5}}^{256}$ is:
	\hfill{\brak{2003}}
	\begin{enumerate}
			\begin{multicols}{4}
		\item $35$
		\item $32$
		\item $33$
		\item $34$
			\end{multicols}{4}
	\end{enumerate}
%\end{enumerate}
%\end{document}
