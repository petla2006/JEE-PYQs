\iffalse
  \title{Assignment}
  \author{Jakkula Adishesh Balaji}
  \section{integer}
\fi

  % \begin{enumerate}
        \item
            The coefficients of three consecutive terms of $\brak{1+x}^{n+5}$ are in the ratio 5:10:14. Then $n=$       
                     \hfill(JEE Adv. 2013)
        \item
            Let m be the smallest positive integer such that the coefficient of $x^{2}$ in the expansion of $\brak{1+x}^{2} + \brak{1+x}^{3} + ... + \brak{1+x}^{49} + \brak{1+x}^{50} + \brak{1+mx}^{50}$ is $\brak{3n+1} \comb{51}{3}$ for some positive integer $n$. Then the value of $n$ is \\
                     \hfill(JEE Adv. 2016)
        \item
            Let \\ $X= \brak{\comb{10}{1}^{2}}+2\brak{\comb{10}{2}^{2}} + 3\brak{\comb{10}{3}^{2}} + ... + 10\brak{\comb{10}{10}^{2}}$, where $\comb{10}{r} , r \in \cbrak{1,2,...,10}$ denote binomial coefficients. Then, the value of $\frac{1}{1430}$X is \rule{10mm}{0.15mm} \\
                    \hfill(JEE Adv. 2018)
        \item
        Suppose \\
           det$\myvec{ \sum\limits_{k=0}^{n}k & \sum\limits_{k=0}^{n} k^{2} \comb{n}{k} \\ \sum\limits_{k=0}^{n}\comb{n}{k} k & \sum\limits_{k=0}^{n}\comb{n}{k} 3^{k}}=0$ \\
holds for some positive integer $n$. The $\sum\limits_{k=0}^{n} \frac{\comb{n}{k}}{k+1}$ equals \rule{10mm}{0.15mm} \\
                    \hfill(JEE Adv. 2019)

  % \end{enumerate}
% \end{document}
