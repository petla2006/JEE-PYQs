\iffalse
\title{assignment}
\author{Tanush Sri Sai Petla}
\section{fitb}
\fi
% \begin{enumerate}
\item If the expression 
$\frac{\sbrak{\sin{\brak{{\frac{x}{2}}}}+\cos{\brak{{\frac{x}{2}}}}+i\tan{\brak{x}}}}{1+2i\sin{\brak{\frac{x}{2}}}}$\hfill\brak{1987 - 2 Marks}\\
is real,then the set of all possible values of $x$ is..... 
\item For any two complex numbers $z_1$,$z_2$ and any real number $a$ and $b$.
$\abs{az_1-bz_2}^2+\abs{bz_1+az_2}^2$ =.....  \hfill\brak{1988 - 2 Marks}
\item If $a$,$b$,$c$ are the numbers between 0 and 1 such that the points $z_1=a+i$, $z_2=1+bi$ and $z_3=0$ form an equilateral triangle,then $a=....$ and $b=.....$ \hfill\brak{1989 - 2 Marks}
\item $ABCD$ is a rhombus. Its diagonals $AC$ and $BD$ intersect at the point M and satisfy $BD=2AC$. If the points D and M represent the complex numbers $1+i$ and $2-i$ respectively, then A represents the complex number..... or..... \hfill\brak{1993 - 2 Marks}
\item Suppose $Z_1$,$Z_2$,$Z_3$ are the vertices of an equilateral triangle inscribed in the circle $\abs{z}=2$.If $Z_1=1+i\sqrt{3}$ then $Z_2=....$ ,$Z_3=.... $ \hfill\brak{1994 - 2 Marks}
\item The value of the expression $1.(2 - \omega)(2 - \omega^3) + 2.(3 - \omega)(3 - \omega^2) + \dots + (n-1)(n - \omega)(n - \omega^2)$ where $\omega$ is an imaginary cube root of unity, is $...$ \hfill\brak{1996 - 2 marks}
% \end{enumerate}

