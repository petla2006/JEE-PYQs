\iffalse
\title{assignment}
\author{Tanush Sri Sai Petla - ai24btech11034}
\section{mcq-single}
\fi
% \begin{enumerate}
\item If the cube roots of unity are 1,$\omega $, $\omega^2$ , then the roots of the equation $\brak{x+1}^8 = 0$ \\
	are
	\hfill\brak{1979}
\begin{multicols}{2}
\begin{enumerate}
\item  $-1$, $i+2\omega$,$1+2\omega^2$
\item  $-1$ ,$1-2\omega$ ,$1-2\omega^2$
\item  $ -1 , -1 ,-1 $ 
\item  None of these
\end{enumerate}
\end{multicols}
\item The smallest positive integer for which
    $\brak{\frac{1+i}{1-i}}^n = 1$ is  \hfill\brak{1980}   
\begin{multicols}{2}
\begin{enumerate}
\item  $n=8$
\item  $n=16$
\item  $n=12$
\item  None of these
\end{enumerate}
\end{multicols}
\item The complex number $z= x+iy$ which satisfy the equation \hfill\brak{1981 - 2 Marks}
     $\abs{\frac{z-5i}{z+5i}} = 1 $lie on 
\begin{multicols}{2}
\begin{enumerate}
\item  the x-axis
\item  the straight line $y=5$
\item  a circle passing through the origin 
\item  None of these
\end{enumerate}
 \end{multicols}
\item If $z=(\frac{\sqrt{3}}{2} + \frac{i}{2})^5$ + $\brak{\frac{\sqrt{3}}{2} - \frac{i}{2}}^5$ , then \hfill\brak{1982 - 2 Marks}
\begin{multicols}{2}
\begin{enumerate}
\item $Re(z)=0$
\item $Im(z)=0$
\item $Re(z)>0, Im(z)>0$
\item $Re(z)>0, Im(z)<0$
\end{enumerate}
\end{multicols}
\item The inequality $\abs{z-4}<\abs{z-2}$ represents the region given by \hfill\brak{1982 - 2 Marks}
\begin{multicols}{2}
\begin{enumerate}
\item  $Re(z)\ge0$
\item  $Re(z)<0$ 
\item  $Re(z)>0$ 
\item  None of these
\end{enumerate}
\end{multicols}
% \end{enumerate}}
