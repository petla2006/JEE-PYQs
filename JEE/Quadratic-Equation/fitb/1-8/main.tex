\iffalse
\title{CHAPTER - 3\\Quadratic equation}
\author{AI24BTECH11028 - Ronit Ranjan}
\section{fitb}
\fi

%\begin{enumerate}
    \item The coefficient of $x^{99}$ in the polynomial \brak{x-1}\brak{x-2}...\brak{x-100}
    is\dots \hfill \brak{1982-2 Marks}
    
    \item If $2+i\sqrt{3}$ is a root of the equation $x^2 + px +q =0$,where $p$ and $q$ are real, then $\brak{p,q}$ = \brak{\dots , \dots}\hfill \brak{1982 - 2 Marks}
    
    \item If the product of the roots of the equation\\
    \begin{align*}
        x^2 -3kx +2e^{2\ln{k}} -1=0
    \end{align*}
    then the roots are real for $k$ = \dots\hfill \brak{1984 - 2 Marks}
    
    \item If the quadratic equation $x^2 + ax +b=0$ and $x^2 + bx + c=0 \brak{a \ne b}$have a common root then value of $a+b$ is \dots \hfill \brak{1986 - 2 Marks}
    
    \item The solution of equation $\log_{7}\log_{5}\brak{\sqrt{x+5}+\sqrt{x}}=0$ is \dots \hfill \brak{1986 - 2 Marks}
    
    \item If $x<0, y,0, x + y + \frac{x}{y} = \frac{1}{2}$ and $\brak{x+y}\brak{\frac{x
    }{y}} = -\frac{1}{2}$, then $x=$ \dots and $y=$ \dots \hfill \brak{1990 - 2 Marks}
    
    \item Let $n$ and $k$ be such positive numbers such that $n \geq \frac{(k)(k+1)}{2}$ . The number of solutions $\brak{x_1,x_2,....x_k}, x_1 \geq 1, x_2 \geq 2,...,x_k \geq k, $ all integers, satisfying $x_1+x_2+...x_k = n$, is \dots  \hfill \brak{1996 - 2Marks}
    \item The sum of all the real roots of the equation $\abs{x-2}^2+\abs{x-2}-2 = 0$ is \hfill \brak{1997 - 2 Marks}
%\end{enumerate}






