\iffalse
\title{CHAPTER - 3\\Quadratic equation}
\author{AI24BTECH11028 - Ronit Ranjan}
\section{true-false}
\fi

%\begin{enumerate}
    \item For every integer $n>1$, the inequality $\brak{n!}^\frac{1}{n} < \frac{n+1}{2}$ holds. \hfill \brak{1981 - 2 Marks}
    \item The equation $2x^2 + 3x + 1 = 0$ has an irrational root. \hfill \brak{1983 - 1 Mark}
    \item If $a<b<c<d$, then the roots of the equation $\brak{x-a}\brak{x-c}+2\brak{x-b}\brak{x-d}=0$ are real and distinct. \hfill \brak{1984 - 1 Mark}
    \item If $n_1, n_2, ....n_p$ are $p$ positive integers, whose sum is an even number, then the number of odd integers among them is odd.\hfill \brak{1985 - 1 Mark}
    \item If $P\brak{x} = ax^2+bx+c$ and $Q\brak{x}= -ax^2+dx+c$, where ac $\neq$ 0, then P\brak{x}Q\brak{x}=0 has at least two real roots. \hfill \brak{1985 - 1 Marks}
    \item If $x$ and $y$ are positive real numbers and $m,n$ are any positive integers, then $\frac{x^ny^m}{\brak{1+ x^{2n}}\brak{1+ y^{2m}}} > \frac{1}{4}$ \hfill \brak{1989 - 1 Mark}



%\end{enumerate}
