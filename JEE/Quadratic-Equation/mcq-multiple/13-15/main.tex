\iffalse
  \title{Assignment}
  \author{Ai24BTECH11024-Pappuri Prahladha}
  \section{mcq-multiple}
\fi


  % \begin{enumerate}
    \item Number of integral divisors of the form $4n+2$\brak{n\geq 0} of the integer 240 is        \hfill (1984-2Marks)
 \begin{multicols}{2}
 \begin{enumerate}
     \item a positive integer
     \item divisible by n
     \item equal to n+$\frac{1}{n}$
     \item never equal to n
 \end{enumerate}
 \end{multicols}
 \item If $3^X=4^x-1$,then $x=$ \hfill (JEE Adv. 2013)
 \begin{multicols}{2}
 \begin{enumerate}
     \item $\frac{2\log_3 2}{2\log_3 2-1}$
     \item $\frac{2}{2-\log_2 3}$
     \item $\frac{1}{1-\log_4 3}$
     \item $\frac{2\log_2 3}{2\log_2 3-1}$
 \end{enumerate}
 \end{multicols}
 \item Let S be the set of all non-zero real numbers \(\alpha\) such that quadratic equation $\alpha x^2-x+\alpha=0$ has two distinct real roots $x_1$ and $x_2$ satisfying the inequality$|x_1-x_2|<1$. Which of the following intervals is(are) $\alpha$ subset of S? \hfill (JEE Adv. 2015)
 \begin{multicols}{2}
 \begin{enumerate}
     \item \brak{-\frac{1}{2},-\frac{1}{\sqrt{5}}}
     \item \brak{-\frac{1}{\sqrt{5}},0}
     \item \brak{0,\frac{1}{\sqrt{5}}}
    \item \brak{\frac{1}{\sqrt{5}},\frac{1}{2}}
 \end{enumerate}
 \end{multicols}


  % \end{enumerate}
% \end{document}
