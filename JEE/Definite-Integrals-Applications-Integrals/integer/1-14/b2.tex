\iffalse
\title{Assignment 1}
\author{AI24BTECH11014 - Charitha Sri}
\section{integer}
\fi

%\begin{enumerate}
\item Let f:R$\rightarrow$ R be a continuous function which 
	satisfies $f(x)=\int_{0}^{x}f\brak{t}dt.$ Then the value of $f\brak{        ln5}$ is \hfill{(2009)}
		
         \item For any real number x, let\sbrak{x} denote the largest integer less than or equal to x. Let f be a real valued function defined on the interval $\sbrak{-10,10}$ by $f\brak{x}$=
		 \[
		 \begin{cases} 
			 $x-\sbrak{x} if \sbrak{x}$ is odd \\ $1+\sbrak{x}-x if \sbrak{x} is even$
                 \end{cases}
	         \]
		Then the value of $\frac{{\pi}^{2}}{10}\int\limits_{-10}^{10}$ $f\brak{x}\cos\pi$ xdx is \hfill{(2011)}


         \item The value of $\int_{0}^{1} 4x^3 \cbrak{\frac{d^2}{dx^2}\brak{1-x^2}^5 }dx$ is \hfill{(JEE Adv. 2014)}

	 \item Let f:R $\rightarrow$ R be a function defined by
		$f(x)=\cbrak{\sbrak{x},x \leq2, 0 if x>2 }$
		where \sbrak{x} is the greatest integer less than or equal to x, if 
		\[ I=\int_{-1}^{2} \frac{xf\brak{x^2}}{2+f\brak{x+1}}dx \],
      then the value of (4I-1) is \hfill{(JEE Adv. 2015)}   
\end{enumerate}
\begin{enumerate}

\item Let $F\brak{x}=\int_{x}^{x^2 +\frac{\pi}{6}} 2\cos^2t(dt)$
	for all x $\in$ R and $f:\sbrak{0,\frac{1}{2}}\rightarrow[0,\infty)$ be a continuous function. For a$\in\sbrak{0,\frac{1}{2}}$, if $F^\prime\brak{a}+2$ is the area of the region bounded by x=0, y=0, $y=f\brak{x}$ and x=a, then $f\brak{0}$ is   \hfill{(JEE Adv. 2015)}

\item If $\alpha=\int_{0}^{1}\brak{e^{9x+3\tan^{-1}x} } \ \brak{ \frac{12+9x^2}{1+x^2}}dx$ where $tan^{-1}x$ takes only principal values,
	then the value of $\brak{\log_e |1+\alpha|-\frac{3\pi}{4}}$ is  \hfill{(JEE Adv. 2015)} 

\item Let f:R$\rightarrow$R be a continuous odd function which vanishes exactly at one point and $f(1)=\frac{1}{2}$. Suppose that $F\brak{x}=\int_{-1}^{x}f\brak{t}dt$ for all $x\in\sbrak{-1,2}$ and $G\brak{x}=\int_{-1}^{x}t|f\brak{f\brak{t}}|dt$ for all $x\in \sbrak{-1,2}$. If $lim_{x\to 1} \frac{F\brak{x}}{G\brak{x}}=\frac{1}{14}$, then the value of $f\brak{\frac{1}{2}}$ is  \hfill{(JEE Adv. 2015)}

\item The total number of distinct $x\in\sbrak{0,1}$ for which $\int_{0}^{x}\frac{t^2}{1+t^4}dt=2x-1$ is \hfill{(JEE Adv. 2016)}

\item Let $f:R\rightarrow$ R be a differentiable function such that $f\brak{0}=0$, $f\brak{\frac{\pi}{2}}=3$ and $f^\prime\brak{0}=1$. If $g\brak{x}=\int_{x}^{\frac{\pi}{2}}\sbrak{f^\prime\brak{t}\cosec t -\cot t\cosec tf\brak{t}}dt$ for $x\in \bigg(0,\frac{\pi}{2}\biggr]$, then $\lim_{x\to 0} g\brak{x}= $   \hfill{(JEE Adv. 2018)}

\item For each positive integer n, let $y_n =\frac{1}{n}\brak{n+1}\brak{n+2}\dots\brak{n+n}^{\frac{1}{n}}$ For $x \in R$, let \sbrak{x} be the greatest integer less than  or equal to x. If $\lim_{n\to \infty} y_n$=L, then the value of [L] is \hfill{(JEE Adv. 2018)}

\item A farmer $F_1$ has a land in the shape of a triangle with vertices at $\vec{P}=\brak{0,0}$,$\vec{Q}=\brak{1,1}$ and $\vec{R}=\brak{2,0}$. From this land, a neighbouring farmer $F_2$ takes away the region which lies between the side PQ and a curve of the form $y=x^n\brak{n>1}$. If the area of the region taken away by the farmer $F_2$ is exactly 30\% of the area of $\Delta$ PQR, then the value of n is \hfill{(JEE Adv. 2018)} 

\item The value of the integral $\int_{0}^{\frac{1}{2}}\frac{1+\sqrt{3}}{\brak{\brak{x+1}^2\brak{1-x}^6}^\frac{1}{4}}dx$ is \hfill{(JEE Adv. 2018)}

\item If $I=\frac{2}{\pi} \int_{\frac{-\pi}{4}}^{\frac{\pi}{4}} \frac{dx}{\brak{1+e^{\sin x}}\brak{2-\cos2x}}$ , then 27 $I^2$ equals \hfill{(JEE Adv. 2019)}

\item The value of the integral $\int_{0}^{\frac{\pi}{2}} \frac{3\sqrt{\cos\theta}}{\brak{\sqrt{\cos\theta}+\sqrt{\sin\theta}}^5} d\theta$ equals \hfill{(JEE Adv. 2019)}
%\end{enumerate}
