\iffalse
  \title{DEFINITE INTEGRALS}
  \author{BADDE VIJAYA SREYAS}
  \section{mains}
\fi

% \begin{enumerate}

%31
	 \item The area of the region bounded by the parabola $\brak{y-2}^2=x-1$, the tangent of the parabola at the point \brak{2, 3} and the $x$-axis is:

		\hfill{(2009)}

		\begin{multicols}{4}
			\begin{enumerate}
				\item 6
				\item 9
				\item 12
				\item 3
			\end{enumerate}
		\end{multicols}

%32
	 \item $\int_0^\pi\sbrak{\cot x}dx$, where $\sbrak{.}$ denotes the greatest integer function, is equal to

		\hfill{(2009)}

		\begin{multicols}{4}
			\begin{enumerate}
				\item 1
				\item -1
				\item $-\frac{\pi}{2}$
				\item $\frac{\pi}{2}$
			\end{enumerate}
		\end{multicols}

%33
	\item The area bounded between the curves $y=\cos x$ and $y=\sin x$ between the ordinates $x=0$ and $x=\frac{3\pi}{2}$ is

		\hfill{(2010)}

		\begin{multicols}{2}
			\begin{enumerate}
				\item $4\sqrt{2}+2$
				\item $4\sqrt{2}-1$
				\item $4\sqrt{2}+1$
				\item $4\sqrt{2}-2$
			\end{enumerate}
		\end{multicols}

%34
	\item Let $p\brak{x}$ be a function defined on \textbf{R} such that $p'\brak{x}=p'\brak{1-x}$, for all $x\in\sbrak{0,1},p\brak{0}=1$ and $p\brak{1}=41$. Then $\int_0^1p\brak{x}dx$ equals
		\hfill{(2010)}

		\begin{multicols}{4}
			\begin{enumerate}
				\item 21
				\item 41
				\item 42
				\item $\sqrt{41}$
			\end{enumerate}
		\end{multicols}

%35
	\item The value of $\int_0^1\frac{8\log\brak{1+x}}{1+x^2}dx$ is

		\hfill{(2011)}

		\begin{multicols}{2}
			\begin{enumerate}
				\item $\frac{\pi}{8}\log2$
				\item $\frac{\pi}{2}\log2$
				\item $\log 2$
				\item $\pi \log2$
			\end{enumerate}
		\end{multicols}

%36
	\item The area of the region enclosed by the curves $y=x, x=e, y=\frac{1}{x}$ and the positive $x$ axis is

		\hfill{(2011)}

		\begin{multicols}{2}
			\begin{enumerate}
				\item 1 square unit
				\item $\frac{3}{2}$ square units
				\item $\frac{5}{2}$ square units
				\item $\frac{1}{2}$ square unit
			\end{enumerate}
		\end{multicols}

%37
	\item The area between the parabolas:$x^2=\frac{y}{4}$ and $x^2=9y$ and the straight line $y=2$ is:
		\hfill{(2012)}

		\begin{multicols}{4}
			\begin{enumerate}
				\item $20\sqrt{2}$
				\item $\frac{10\sqrt{2}}{3}$
				\item $\frac{20\sqrt{2}}{3}$
				\item $10\sqrt{2}$
			\end{enumerate}
		\end{multicols}

%38
	\item If $g(x)=\int_0^x\cos 4t dt$, then $g\brak{x+\pi}$ equals

		\hfill{(2012)}

		\begin{multicols}{2}
			\begin{enumerate}[label=(\alph*)]
				\item $\frac{g(x)}{g\brak{\pi}}$
				\item $g(x)+g\brak{\pi}$
				\item $g(x)-g\brak{\pi}$
				\item $g(x).g\brak{\pi}$
			\end{enumerate}
		\end{multicols}

%39
	\item \textbf{Statement-1 :} The value of the integral $\int_{\pi/6}^{\pi/3}\frac{dx}{1+\sqrt{\tan x}}$ is equal to $\pi/6$

		\textbf{Statement-2 :} $\int_a^bf\brak{x}dx=\int_a^bf\brak{a+b-x}dx$.

		\hfill{(JEE M 2013)}
		
		(a) Statement-1 is true; Statement-2 is true; Statement-2 is a correct explanation for Statement-1

		(b) Statement-1 is true; Statement-2 is true; Statement-2 is not a correct explanation for Statement-1

		(c) Statement-1 is true; Statement-2 is false

		(d) Statement-1 is false; Statement-2 is true

%40
	\item The area (in square units) bounded by the curves $y=\sqrt{x}$, $2y-x+3=0$, $x$-axis, and lying in the first quadrant is :
		\hfill{(JEE M 2013)}

		\begin{multicols}{4}
			\begin{enumerate}
		\item 9
		\item 36
		\item 18
		\item $\frac{27}{4}$
			\end{enumerate}
		\end{multicols}


%41
	\item The integral $\int_0^{\pi}\sqrt{1+4\sin ^2\frac{x}{2}-4\sin \frac{x}{2}}dx$ equals:

		\hfill{(JEE M 2014)}

		\begin{multicols}{2}
			\begin{enumerate}
				\item (a) $4\sqrt{3}-4$
				\item (b) $4\sqrt{3}-4-\frac{\pi}{3}$
				\item (c) $\pi-4$
				\item (d) $\frac{2\pi}{3}-4-4\sqrt{3}$
			\end{enumerate}
		\end{multicols}

%42
	\item The area of the region described by $A=\{\brak{x,y}:x^2+y^2\leq1$ and $y^2\leq1-x\}$ is:

		\hfill{(JEE M 2014)}

		\begin{multicols}{4}
			\begin{enumerate}
				\item $\frac{\pi}{2}-\frac{2}{3}$
				\item $\frac{\pi}{2}+\frac{2}{3}$
				\item $\frac{\pi}{2}+\frac{4}{3}$
				\item $\frac{\pi}{2}-\frac{4}{3}$
			\end{enumerate}
		\end{multicols}

%43
	\item The area (in sq. units) of the region described by $\{\brak{x,y}:y^2\leq2x$ and $y\geq4x-1\}$ is

		\hfill{(JEE M 2015)}

		\begin{multicols}{4}
			\begin{enumerate}
				\item $\frac{15}{64}$
				\item $\frac{9}{32}$
				\item $\frac{7}{32}$
				\item $\frac{5}{64}$
			\end{enumerate}
		\end{multicols}

%44
	\item The integral $\int_2^4\frac{\log x^2}{\log x^2 + \log\brak{36-12x+x^2}}dx$ is equal to:

		\hfill{(JEE M 2015)}

		\begin{multicols}{4}
			\begin{enumerate}
				\item 1
				\item 6
				\item 2
				\item 4
			\end{enumerate}
		\end{multicols}

%45
	\item The area (in sq. units) of the region $\{\brak{x,y}:y^2\geq2x$ and $x^2+y^2\leq4x, x\geq0, y\geq0\}$ is

		\hfill{(JEE M 2016)}

		\begin{multicols}{2}
			\begin{enumerate}
				\item $\pi-\frac{4\sqrt{2}}{3}$
				\item $\frac{\pi}{2}-\frac{2\sqrt{2}}{3}$
				\item $\pi-\frac{4}{3}$
				\item $\pi-\frac{8}{3}$ 
			\end{enumerate}
		\end{multicols} 
% \end{emumerate}
