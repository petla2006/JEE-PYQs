\iffalse
\let\negmedspace\undefined
\let\negthickspace\undefined
\documentclass[journal]{IEEEtran}
\usepackage[a5paper, margin=10mm, onecolumn]{geometry}
%\usepackage{lmodern} % Ensure lmodern is loaded for pdflatex
\usepackage{tfrupee} % Include tfrupee package

\setlength{\headheight}{1cm} % Set the height of the header box
\setlength{\headsep}{0mm}     % Set the distance between the header box and the top of the text

\usepackage{gvv-book}
\usepackage{gvv}
\usepackage{cite}
\usepackage{amsmath,amssymb,amsfonts,amsthm}
\usepackage{algorithmic}
\usepackage{graphicx}
\usepackage{textcomp}
\usepackage{xcolor}
\usepackage{txfonts}
\usepackage{listings}
\usepackage{enumitem}
\usepackage{mathtools}
\usepackage{gensymb}
\usepackage{comment}
\usepackage[breaklinks=true]{hyperref}
\usepackage{tkz-euclide} 
\usepackage{listings}
% \usepackage{gvv}                                        
\def\inputGnumericTable{}                                 
\usepackage[latin1]{inputenc}                                
\usepackage{color}                                            
\usepackage{array}                                            
\usepackage{longtable}                                       
\usepackage{calc}                                             
\usepackage{multirow}                                         
\usepackage{hhline}                                           
\usepackage{ifthen}                                           
\usepackage{lscape}
\bibliographystyle{IEEEtran}
\vspace{3cm}

\title{CHAPTER - 20\\Vector Algebra and\\Three Dimensional Geometry}
\author{EE24BTECH11061 - Rohith Sai}
\maketitle

\renewcommand{\thefigure}{\theenumi}
\renewcommand{\thetable}{\theenumi}

\section{true-false}
\fi
%\begin{enumerate}
\item Let $\vec{A}$, $\vec{B}$ and $\vec{C}$ be unit vectors suppose that $\vec{A}.\vec{B} = \vec{A}.\vec{C}=0$, and that the angle between $\vec{B}$ and $\vec{C}$ is $\frac{\pi}{6}$. Then $\vec{A}=\pm2\brak{\vec{B}\times\vec{C}}$.
\hfill (1981 - 2 Marks)

\item If $\vec{X}.\vec{A}=0$ , $\vec{X}.\vec{B}=0$, $\vec{X}.\vec{C}=0$ for some non-zero vector $\vec{X}$, then $\sbrak{\vec{A}\ \vec{B}\ \vec{C}}=0$\\
\hfill (1983 - 1 Mark)

\item The points with position vectors $\vec{a+b}$, $\vec{a-b}$ and $\vec{a+kb}$ are collinear for all real values of $\vec{k}$.
\hfill (1984 - 1 Mark)

\item For any three vectors $\vec{a}, \vec{b}$ and $\vec{c}$, $\brak{\vec{a}-\vec{b}}.\brak{{\brak{\vec{b}-\vec{c)}\times(\vec{c}-\vec{a}}}} = 2\vec{a}.(\vec{b}\times\vec{c}).$\\
\hfill (1989 - 1 Mark)
%\end{enumerate}
