\iffalse
\title{Assignment}
\author{M.Ranjith}
\section{fitb}
\fi
\item The values of \begin{align*}f\brak{x}=3\sin\brak{\sqrt{\frac{\pi^2}{16}-x^2}}\end{align*}lie in the interval \dots 
    
    
    \hfill{(1983 - 1 Mark)}
    
    
     \item For the function \begin{align*}f\brak{x} = \begin{cases} \frac{x}{1 + e^{1/x}}, & x \neq 0 \\ 0, & x = 0 \end{cases} \end{align*}
    the derivative from the right, $f^{\prime}\brak{0+} $= \dots , and the derivative from the left, $f^{\prime }\brak{0-}$= \dots
    
    
     
    \hfill{(1983 - 2mark)}
    
    \item{The domain of the function $f\brak{x}=\sin^{-1}\brak{\log_{2}\brak{\frac{x^{2}}{2}}}$ is given by \dots
    
    
    \hfill 
    {(1984 - 2mark)}
    
    \item Let $A$ be a set of $n$ distinct elements. Then the total number of distinct functions from $ A $ to $ A $ is \dots  and out of these \dots are onto functions.
    
    \hfill
    {(1985- 2mark)}
    
    
    \item If \begin{align*}  f\brak{x} = \sin \brak{ \ln \brak{ \frac{\sqrt{4 - x^{2}}}{1 - x}}} \end{align*},  then domain of  $f\brak{x}$ is ... and its range is \dots
    
    
    \hfill
    {(1985 - 2Mark)}
    
     
    \item There are exactly two distinct linear functions, \dots and \dots which map $[-1,1]onto [0,2]$
    
    \hfill
    {(1989 - 1Mark)}
    
    
    
     \item If $f$ is a even function defined on the 
    interval $\brak{-5,5}$,then four real values of $x $
    satisfying the equation $f\brak{x}=f\brak{{\frac{x+2}{x+1}}}$
    are \dots  and \dots 
    
    
    \hfill   (1996 - 1mark)
    }
