\iffalse
\title{Matrices and Determinants}
\author{EE24BTECH11001 -  ADITYA TRIPATHY}
\section{mcq-single}
\fi
\item If $ f\brak{x} = $
    \begin{align*}
        \mydet{
            1 & x & x+1 \\
            2x & x\brak{x-1} & \brak{x+1}x \\
            3x\brak{x-1} & x\brak{x-1}\brak{x-2} & \brak{x+1}x\brak{x-1}
    } \end{align*} then $f\brak{100}$ is equal to 
    \hfill{\brak{1999-2Marks}}
    \begin{multicols}{4}
        \begin{enumerate}
            \item 0 
                \columnbreak
            \item 1
                \columnbreak
            \item 100
                \columnbreak
            \item -100
        \end{enumerate}
    \end{multicols}

\item If the system of equations
    \begin{align*}
        x-ky-z=0 ,\\ kx-y-z=0,x+y-z=0
    \end{align*} has a non-zero  solution,  then the possible values of k are 
    \hfill{\brak{2000S}}
    \begin{multicols}{4}
        \begin{enumerate}
            \item -1,2 \columnbreak
            \item 1,2 \columnbreak
            \item 0,1 \columnbreak
            \item -1,1
        \end{enumerate}
    \end{multicols}


\item Let $\omega = -\frac{1}{2} +i\frac{\sqrt{3}}{2}$. Then the value of the determinant \hfill{\brak{2002S}}
    \begin{align*}
        \mydet{
            1 & 1 & 1 \\
            1 & -1-\omega^2 & \omega^2 \\
            1 & \omega^2 & \omega^4 
        }
    \end{align*}
    \begin{enumerate}
            \begin{multicols}{2}
            \item $3\omega$ \columnbreak
            \item $3\omega\brak{\omega-1}$
            \end{multicols}
            \begin{multicols}{2}
            \item $3\omega^2$ \columnbreak
            \item $3\omega\brak{1-\omega}$
            \end{multicols}
    \end{enumerate}


\item The number of values of $k$ for which the system of equations 
    \begin{align*}
    \brak{k+1}x + 8y=4k; \\ kx +\brak{k+3}y=3k-1 \end{align*} has infinitely many solutions is 
    \hfill{\brak{2002S}}
    \begin{enumerate}
            \begin{multicols}{4}
            \item 0 \columnbreak
            \item 1 \columnbreak
            \item 2 \columnbreak
            \item infinte
            \end{multicols}
    \end{enumerate}

\item If $A=$
    \myvec{
        \alpha & 0 \\
        1 & 1
    } and $B=$ \myvec{
        1 & 0 \\
        5 & 1
    }, then value of $\alpha$ for which $A^2 = B$,is
    \hfill{\brak{2003S}}
    \begin{enumerate}
            \begin{multicols}{2}
            \item 1 \columnbreak
            \item 4 
            \end{multicols}
            \begin{multicols}{2}

            \item 2 \columnbreak
            \item infinite
            \end{multicols}
    \end{enumerate}


\item If the system of equations $x + ay = 0, az + y =0$ and $ax + z =0$ has infinite solutions, then the value of $a$ is 
    \hfill{\brak{2003S}}
    \begin{enumerate}
            \begin{multicols}{2}
            \item -1 \columnbreak
            \item 1
            \end{multicols}
            \begin{multicols}{2}

            \item 0 \columnbreak
            \item no real values
            \end{multicols}
    \end{enumerate}


\item  Given \begin{align*} 2x-y+2z=2,\\x-2y+z=-4,\\x+y+\lambda z=4 \end{align*} then the value of $\lambda$ such that the given system of equation has NO solution, is

        \hfill{\brak{2004S}}
        \begin{enumerate}
                \begin{multicols}{4}
                \item 3 \columnbreak
                \item 1 \columnbreak
                \item 0 \columnbreak
                \item -3
                \end{multicols}
        \end{enumerate}
    \item Is $A=$ \myvec{
            \alpha & 2\\
            2 & \alpha
        } and $\mydet{A^3}=125$ then the value $\alpha$ is
        \hfill{\brak{2004S}}
        \begin{enumerate}
                \begin{multicols}{4}
                \item $\pm 1$ \columnbreak
                \item $\pm 2$ \columnbreak
                \item $\pm 3$ \columnbreak
                \item $\pm 5$
                \end{multicols}
        \end{enumerate}


    \item $A=$ \myvec{1 & 0 & 0 \\ 0 & 1&1 \\ 0 &-2 &4} and $I=$ \myvec{1 & 0 &0\\ 0 & 1 & 0\\0 & 0 & 1} \hfill{2005S}
        and $A^{-1} = $ \myvec{\frac{1}{6}\brak{A^2 + cA +dI}}, then the value of $c$ and $d$ are \hfill{\brak{2005S}}
        \begin{enumerate}
                \begin{multicols}{2}
                \item\brak{-6,-11} \columnbreak
                \item\brak{6,11}
                \end{multicols}
                \begin{multicols}{2}
                \item\brak{-6,11} \columnbreak
                \item\brak{6,-11}
                \end{multicols}
        \end{enumerate}


    \item If $P=$ 
        \myvec{
            \frac{\sqrt{3}}{2} & \frac{1}{2}\\
            -\frac{1}{2} & \frac{\sqrt{3}}{2}
        }and $A = \begin{vmatrix} 1& 1 \\ 0 & 1\end{vmatrix}$ and $Q = PAP^T$ and $x=P^{T}Q^{2005}P$ then $x$ is equal to 


            \begin{enumerate}
                \item \mydet{ 1 & 2005\\0 & 1 }
                \item $\mydet{ 4 + 2005\sqrt{3} & 6015 \\ 2005 & 4 - 2005\sqrt{3}}$
                \item $\frac{1}{4}\mydet{2 + \sqrt{3} & 1 \\ -1 & 2 -\sqrt{3}}$
                \item $\frac{1}{4}\mydet{2005 & 2 - \sqrt{3} \\ 2 + \sqrt{3} & 2005}$
            \end{enumerate}		
        \item Consider 3 points 
            \begin{align*}
                P=\brak{-\sin \brak{\beta - \alpha}, - \cos\beta}, Q = \brak{\cos \brak{\beta - \alpha}, \sin\beta}
            \end{align*} and 
            \begin{align*} R = \brak{\cos \brak{\beta - \alpha + \theta}, \sin\brak{\beta - \theta}} \end{align*} where $0<\alpha,\beta,\theta<\frac{\pi}{4}.$Then, \hfill{\brak{2008}}
                \begin{enumerate}
                    \item $P$ lies on the same segment $RQ$
                    \item $Q$ lies on the line segment $PR$
                    \item $R$ lies on the line segment $QP$
                    \item $P$,$Q$,$R$ are non-collinear
                \end{enumerate}
            \item The number of 3x3 matrices $A$ whose entries are either $0$ or $1$ and for which the system $A\myvec{x\\y\\z}=\myvec{1\\0\\0}$ has exactly two distinct solutions is \hfill{\brak{2008}}

                \begin{enumerate}
                        \begin{multicols}{4}
                        \item 0 \columnbreak
                        \item $2^9 - 1$ \columnbreak
                        \item 168 \columnbreak
                        \item 2
                        \end{multicols}
                \end{enumerate}
            \item Let $\omega \neq 1$ be a cube root of unity and $S$ be the set of all non-singular matrices of the form 
                \begin{align*}
                    \mydet{
                        1 & a & b \\
                        \omega & 1 & c\\
                        \omega^2 & \omega & 1
                    }
                \end{align*} where each of $a,b$ and $c$ is either $\omega$ or $\omega^2$. Then the number of distinct matrices in the set $S$ is
                \hfill{\brak{2008}}

                \begin{enumerate}

                        \begin{multicols}{4}
                        \item 2\columnbreak
                        \item 6\columnbreak
                        \item 4\columnbreak
                        \item 8
                        \end{multicols}
                \end{enumerate}

            \item Let $P = \myvec{a_{ij}}$ be 3x3 matrix and let $Q =  \myvec{b_{ij}}$, where $b_{ij} = 2^{i+j}a_{ij}$ for $1 \le i,j \le 3$. If the determinant of $P$ is 2, then the determinant of the matrix $Q$ is 
                \hfill{\brak{2012}}
                \begin{enumerate}

                        \begin{multicols}{4}
                        \item $2^{10}$ \columnbreak
                        \item $2^{11}$ \columnbreak
                        \item $2^{12}$\columnbreak
                        \item $2^{13}$
                        \end{multicols}
                \end{enumerate}

            \item If $P$ is a 3x3 matrix such that $P^T = 2P +I$, where $P^T$ is the transpose of $P$ and $I$ is the 3x3 identity matrix, then there exists a column matrix $X=\myvec{x\\y\\z} \neq \myvec{0\\0\\0}$
                \hfill{\brak{2012}}
                \begin{enumerate}

                        \begin{multicols}{2}
                        \item $PX=\myvec{0\\0\\0}$ \columnbreak
                        \item $PX=X$
                        \end{multicols}
                        \begin{multicols}{2}
                        \item $PX=2X$ \columnbreak
                        \item $PX = -X$
                        \end{multicols}
                \end{enumerate}

            \item Let $P=\myvec{1&0&0\\4&1&0\\16&4&1}$ and $I$ be the identity matrix of order 3. If $Q = \myvec{q_{ij}}$ is a matrix such that $P^{50} -Q =I$, then $\frac{q_{31}+q_{32}}{q_{21}}$ equals
                \hfill{\brak{JEE Adv. 2016}}
                \begin{enumerate}

                        \begin{multicols}{4}
                        \item52 \columnbreak
                        \item103 \columnbreak
                        \item201 \columnbreak
                        \item205 
                        \end{multicols}
                \end{enumerate}
