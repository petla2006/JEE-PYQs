\iffalse
  \title{conic-sections}
  \author{KOTHAPALLI AKHIL}
  \section{paragraph}
\fi
%\begin{enumerate}
\item  The equation of the locus of the point whose distances from the point $\Vec{P}$ and the line $AB$ are equal, is
\end{enumerate}

\begin{enumerate}
     \item $9x^2+y^2-6xy-54x-62y+241=0$
     \item $x^2+9y^2+6xy-54x-62y-241=0$
     \item $9x^2+9y^2-6xy-54x-62y-241=0$
     \item $x^2+y^2-2xy+27x+31y-120=0$
\end{enumerate}
\subsection*{Passage 4} 
\begin{enumerate}
\item[] Let $PQ$ be a focal chord of the parabola $y^2=4ax$.The tangents to the parabola at $\Vec{P}$ and $\Vec{Q}$ meet at a point lying on the line $y=2x+a$,$a>0$
\item Length of the chord $PQ$ is

\hfill(JEE Adv.2013)        
\begin{enumerate}
    \item $7a$
    \item $5a$
    \item $2a$
    \item $3a$
\end{enumerate}

\item If chord $PQ$ subtends an angle $\theta$  at the vertex of $y^2=4ax$
\hfill(JEE Adv.2013)

\begin{enumerate}
    \item $\frac{2}{3}\sqrt{7}$
    
    \item $\frac{-2}{3}\sqrt{7}$
    
    \item $\frac{2}{3}\sqrt{5}$
    
    \item $\frac{-2}{3}\sqrt{5}$
\end{enumerate}
\end{enumerate}
\subsection*{Passage 5}
\begin{enumerate}
\item[] Let $a$,$r$,$s$,$t$ be nonzero real numbers. Lets $\Vec{P}$$(at^2,2as)$,$\Vec{Q}$,$\Vec{R}$$(as^2,2as)$ be distinct points on the parabola $y^2=4ax$.suppose that PQ is the focal chord and lines $QR$ and $PK$ are parallel,where $K$ is the point $(2a,0)$
\end{enumerate}
\begin{enumerate}
\item The value of $r$ is 
\hfill(JEE Adv.2014)
\begin{enumerate}
    \item $\frac{-1}{t}$ 
    \item $\frac{t^2+1}{t}$
    \item $\frac{1}{t}$
    \item $\frac{t^2-1}{t}$
\end{enumerate}
\item If $st=1$, then the tangent at $\Vec{P}$ and the normal at $\Vec{M}$ to the
parabola meet at a point whose ordinate is 
\begin{enumerate}
    \item $\frac{a(t^2+1)^2}{t^3}$
    \item $\frac{a(t^2+1)}{2t^3}$
    \item $\frac{1}{t}$
    \item $\frac{t^2-1}{t}$
    \end{enumerate}
\end{enumerate}
\subsection*{Passage 6}
\begin{enumerate}
\item[] Let $\vec{F_1}$($x_1$,0) and  $\vec{F_2}$($x_2$,0) for $x_1<0$ and $x_2>0$, be the focii of the ellipse $\frac{x^2}{9}+\frac{y^2}{8} =1$. Suppose a parabola having vertex at the origin and focus at $\vec{F_2}$ intersects the ellipse at point $\Vec{M}$ in the first quadrant and the point $\Vec{N}$ in the first quadrant.
\item The orthocentre of the triangle $F_1$MN is

         \hfill(JEE Adv. 2016)
\begin{enumerate}
     \item $(\frac{-9}{10},0)$   
     \item $(\frac{2}{3},0)$     
     \item $($9/10$,0)$ 
    \item $(\frac{2}{3},\sqrt{6})$ 
    \end{enumerate}
\item If the tangents to the ellipse at $\Vec{M}$ and $\Vec{N}$ meet at $\Vec{R}$ and the normal to the parabola at $\Vec{M}$ meets the X-Axis at $\Vec{Q}$, the the ratio of area of the triangle $MQR$ to the area of the quadrilateral M$F_1$N$F_2$ is
\hfill(JEE Adv.2016)
\begin{enumerate}
    \item $3:4$
    \item $4:5$
    \item $5:8$
    \item $2:3$
\end{enumerate}
%\end{enumerate}

% \end{document}
