\iffalse

\title{Properties of Triangle}
\author{ee24btech11051 - Prajwal}
\section{fitb}
\fi
%\begin{enumerate}
    \item In a $\Delta ABC$, $\angle A=90\degree$ and $AD$ is an altitude. Complete the relation\\
    $\frac{BD}{BA} = \frac{AB}{(\dots)}$
    \hfill (1980)
    
    \item $ABC$ is a triangle, $P$ is a point on $AB$, and $Q$ is point on $AC$ such that $\angle AQP = \angle ABC$. Complete the relation
    $\frac{area\ of \Delta APQ}{area\ of \Delta ABC} =\frac{(\dots)}{AC^2}$
    \hfill (1980)
    
    \item $ABC$ is a triangle with $\angle B $ greater than $\angle C$ 
    $D$ and $E$ are the points on $BC$ such that $AD$ is perpendicular to $BC$ and $AE$ is the bisector of angle $A$ .Complete the relation
    $\angle DAE = \frac{1}{2} [( ) - \angle C]$
    \hfill (1980)
    \item the set of all real numbers $a$ such that $a^2 + 2a, 2a + 3$ and $a^2 + 3a + 8$ are the sides of a triangle is \dots
    \hfill (1985 - 2 Marks)
    \item In a triangle $ABC$, if $\cot A$,$\cot B$,$\cot C$ are in A.P. ,then $a^2$,$b^2$,$c^2$,are in \dots progression \hfill (1985 - 2 Marks)
    \item A polygon of nine sides, each of length $2$, is inscribed in a circle. The radius of the circle is \dots \hfill (1987 - 2 Marks) 
    \item If the angles of a triangle are $30\degree$ and $45\degree$ and the included side is $(\sqrt{3} + 1) cms$, then the area of the triangle is \dots \hfill (1988 - 2 Marks)
    \item If the triangle $ABC$, $\frac{2\cos A}{a} + \frac{2\cos B}{b} + \frac{2\cos C}{c} = \frac{a}{bc} +  \frac{b}{ac}$, then the value of the angle $A$ is \dots degrees. \hfil (1993 - 2 Marks)
    \item In the triangle $ABC$, $AD$ is the altitude from $A$. Given $b>c$, $\angle C=23 \degree$ and $AD = \frac{abc}{b^2 - c^2}$ then $\angle B = $ \dots \hfill (1994 - 2 Marks)
    \item A circle is inscribed in a equilateral triangle of a side $a$. The area of any square inscribed in this circle is \dots \hfill (1994 - 2 Marks)  
    \item In a triangle $ABC$, $a:b:c = 4:5:6$. The ratio of the radius of the circumstances to that of the incircle is \dots \hfill (1996 - 1 Marks) 
%\end{enumerate}



%\end{document}
