\iffalse
\title{Assignment}
\author{M.Ranjith}
\section{mcq-single}
\fi
\item The differential equation whose solution is $Ax^2 + By^2 = 1$ where $A$ and $B$ are arbitrary constants is of 



\begin{multicols}{2}
\begin{enumerate}
    \item second order and second degree 
    \item first order and second degree 
    \item first order and first degree 
    \item second order and first degree 

\end{enumerate}
\end{multicols}
\hfill
{[2006]}
\item The differential equations of all the circles passing through the origin and having their centres on x-axis is
\begin{multicols}{2}
\begin{enumerate}
\item $ y^2=x^2 + 2xy\diff{y}{x} $
\item $ y^2=x^2 - 2xy\diff{y}{x}$
\item $ x^2=y^2 + xy\diff{y}{x}$
\item $ x^2 =y^2 + 3xy\diff{y}{x}$
\end{enumerate}
\end{multicols}
\hfill
{{[2007]}}




\item The solution of the differential equation $ \diff{y}{x}=\frac{x+y}{x}$ satisfying the condition $ y\brak{1}=1 $ is
  \begin{multicols}{2}
    \begin{enumerate}
    \item $ y=  \ln{x}+x $
    \item $y=x\ln{x}+x^2$
    \item $ y=xe^{\brak{x-1}} $
    \item $ y=x \ln{x}+ x$
       \end{enumerate}
   \end{multicols}
\hfill
{[2008]}


\item The differential equation which represents the family of curves $y= c{_1}e^{c_2}x$, where $c{_1}$ and $c{_2}$ are arbitrary constants, is
\begin{multicols}{2}
 \begin{enumerate}
    \item $\frac{d^{2}y}{dx^{2}}$= y\diff{y}{x}

    \item $y\frac{d^{2}y}{dx^{2}}$=\diff{y}{x} 

    \item $y\frac{d^{2}y}{dx^{2}}=\brak{\diff{y}{x}}^2$
    
    \item $\diff{y}{x}=y^2$
 \end{enumerate}
\end{multicols}
\hfill
{[2009]}
\item Solutions of the differential equation $\cos{x} d{y}=y(\sin{x}-y)d{x},0<x<\frac{\pi}{2}$ is
\begin{multicols}{2}
 \begin{enumerate}
    
    \item $ y\sec{x}=\tan{x}+c$
    \item $y\tan{x}=\sec{x}+c$
    \item $\tan{x}=(\sec{x}+c)y$
    \item $\sec{x}=(\tan{x}+c)y$
\hfill
{{[2010]}}




    
 \end{enumerate}
\end{multicols}

\item If $\frac{d^{2}y}{dx^{2}}=y+3$ and $y\brak{0}=2$, then $y\brak{\ln{2}}$ is equal to:
\begin{multicols}{2}
 \begin{enumerate}
    \item $ 5 $
    \item $ 13 $
    \item $ -2 $
    \item $ 7 $.
 \end {enumerate}
\end{multicols}
\hfill
{{[2011]}}



\item Let be the purchase value of an equipment and $V\brak{t}$ be the value after it has been used for t years. The value V\brak{t} depreciates at a rate given by differential equation \diff{V_{\brak{t}}}{t}=-k\brak{T-t}. where k is a constant and T is the total life in years of the equipment. Then the scrap value $V\brak{T}$ of the equipment is 
\begin{multicols}{2}
 \begin {enumerate}

    \item $ l -\frac{kT^2}{2}$
    \item $ l - \frac{k\brak{T-t}^2}{2}$
    \item $ e^{-kT}$
    \item $ T^2-\frac{1}{k}$

 \end{enumerate}
\end{multicols}
\hfill
{[2011]}

\item The population $p\brak{t}$ at time of a certain mouse species satisfies the differential equation $\diff{p\brak{t}}{t}= 0.5p\brak{t}- 450$. If $p\brak{0}=850$,then the time at which the population becomes zero is:
\begin{multicols}{2}
 \begin{enumerate}
    \item $ 2\ln{18}$
    \item $ 2\ln{9}$
    \item $ \frac{1}{2}\ln{18}$
 \end{enumerate}
\end{multicols}
\hfill
{[2012]}

\item At present, a firm is manufacturing $2000$ times. It is estimated that the rate of change of production $P$ with respect to additional number of workers $x$ is given by $\diff{P}{x}=100-12\sqrt{x}$. If the firm employs 25 more workers, then the new level of production of items is

\begin{multicols}{2}
    

 \begin{enumerate}

    \item $2500$
    \item$3000$
    \item$3500$
    \item$4500$
 \end{enumerate}
\end{multicols}

\hfill
{[JEE M 2013]}


\item Let the population of rabbits surviving at time t be governed by the differential equation $\diff{p\brak{t}}{t}$=$\frac{1}{2}p\brak{t}-200$. If P\brak{0}=100, then p\brak{t} equals:

\begin{multicols}{2}
 \begin{enumerate}
    \item $ 600-500e^\frac{t}{2}$
    \item $ 400-300e^\frac{-t}{2}$
    \item $ 400-300e^\frac{t}{2}$
    \item $ 300-200e^\frac{-t}{2}$
 \end{enumerate}
\end{multicols}
\hfill
{[JEE M 2014]}
\item Let $y\brak{x}$ be the solution of the differential equation $\brak{x\log{x}}\diff{y}{x} + y = 2x \log{x},\brak{x\geq 1}$. Then $y\brak{e}$ is equal to:
\begin{multicols}{2}
 \begin{enumerate}
    \item $ 2 $
    \item $ 2e $
    \item $ e $
    \item $ 0 $
 \end{enumerate}
\end{multicols}
\hfill
{[JEE M 2015]}

\item If the curve $ y=f\brak{x}$ passes through the point $ \brak{1,1}$ and satisfies the differential equation, $ y\brak{1+xy}d{x}=xd{y}$, then $ f\brak{\frac{-1}{2}}$ is equals to 
\begin{multicols}{2}
 \begin{enumerate}
    \item $ \frac{2}{5}$



    \item $ \frac{4}{5}$


    \item $ \frac{2}{5}$



    \item $ \frac{4}{5}$


  \end{enumerate}
\end{multicols}
\hfill
{[JEE M 2016]}

\item If $ \brak{24 \sin{x}}\diff{y}{x}+ \brak{y+1}\cos{x}=0$ and $y\brak{0}=1$ then $y\brak{\frac{\pi}{2}}$ is equal to
\begin{multicols}{2}
 \begin{enumerate}
    \item $\frac{4}{3}$
    \item $\frac{1}{3}$
    \item $\frac{2}{3}$
    \item $\frac{1}{3}$
    
 \end{enumerate}
\end{multicols}
\hfill
{[JEE M 2017]}


\item Let $ y=y\brak{x}$ be the solution of the differential equation $\sin{x}\diff{y}{x}+y\cos{x}=4x$, $x\in\brak{0,2}$. If $y\brak{\frac{\pi}{2}}=0$, then $y\brak{\frac{\pi}{6}}$ is equal to
\begin{multicols}{2}
 \begin{enumerate}
    \item $ \frac{-8}{9\sqrt{3}}\pi^2$
    \item $ \frac{-8}{9}\pi^2$
    \item $\frac{-4}{9}\pi^2$
    \item $ \frac{4}{9\sqrt{3}}\pi^2 $
 \end{enumerate}
\end{multicols}
\hfill
{[JEE M 2018]}
\item If $ y=y\brak{x}$ is the differential equation $ \sin{x}\diff{y}{x}+2y=x^2$ satisfying $y\brak{a}=1$,then $y\brak{\frac{1}{2}}$ is equal to
\begin{multicols}{2}
 \begin{enumerate}
    \item $ \frac{7}{64}$

    
    \item $ \frac{1}{4}$

    
    \item $ \frac{49}{16}$
    
    \item $ \frac{13}{16}$
    
    
 \end{enumerate}
\end{multicols}
\hfill
{[JEE M 2019-9April(M)]}
\item The solution of the differential equation $ x\diff{y}{x}+2y=x^2 \brak{x\neq0}$with$ y\brak{1}=1$, is:
\begin{multicols}{2}
 \begin{enumerate}
    \item $ y=\frac{4}{5}x^3+\frac{1}{5x^2}$

    \item $ y=\frac{x^3}{5}+\frac{1}{5x^2}$
    \item $ y=\frac{x^2}{4}+\frac{3}{4x^2}$
    \item $ y=\frac{3}{4}x^2+\frac{1}{4x^2}$
  \end{enumerate}
\end{multicols}
\hfill
{[JEE M 2019-9April(M)]}


