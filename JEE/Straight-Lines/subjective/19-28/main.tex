\iffalse
\title{Assignment-2 }
\author{AI24BTECH11036- Shreedhanvi Yadlapally}
\section{subjective}
\fi
% \begin{enumerate}

	\item Tangent at a point $\vec{P_{1}}$ {other than \brak{0, 0}} on the curve $y=x^{3}$ meets the curve again at $\vec{P_{2}}$. The tangent at $\vec{P_{2}}$ meets the curve again at $\vec{P_{1}}$, and so on. Show that the abscissae of $\vec{P_{1}, P_{2}, P_{3} \dots P_{n}}$, form a G.P. Also find the ratio. $\frac{\sbrak{area(\Delta P_{1}P_{2}P_{3})}}{\sbrak{area(\Delta P_{2}P_{3}P_{4})}}$

	\hfill{(1993 - 5 Marks)}

\item A line through $\vec{A} \brak{5, 4}$ meets the line $x+3y+2=0$, $2x+y+4=0$ and $x-y-5=0$ at the points $\vec{B}$, $\vec{C}$ and $\vec{D}$ respectively. If $\brak{15/AB}^{2}+\brak{10/AC}^{2}-\brak{6/AD}^{2}$, find the equation of the line.

	\hfill{(1993 - 5 Marks)}

\item A rectangle $PQRS$ has it side $PQ$ parallel to the line $y=mx$ 
	and vertices $\vec{P}$, $\vec{Q}$ and $\vec{S}$ on the lines $y=a$, $x=b$ and $x=-b$, 
		respectively. Find the locus of the vertex $\vec{R}$.
	
		\hfill{(1996 - 2 Marks)}

\item Using co-ordinate geometry, prove that the three altitudes of any triangle are concurrent.

	\hfill{(1998 - 8 Marks)}

\item For points $\vec{P}=\brak{x_{1}, y_{1}}$ and $\vec{Q}=\brak{x_{2}, y_{2}}$ of the co-ordn=inate plane, a new distance $d\brak{P, Q}$ is defined by $d\brak{P, Q}=\abs{x_{1}-x_{2}}+\abs{y_{1}-y_{2}}$. Let $\vec{O}=\brak{0, 0}$ and $\vec{A}=\brak{3, 2}$. Prove that the set of points in the first quadrant which are equidistant (with respect to the new distance) from $\vec{O}$ and $\vec{A}$ consists of the union of a line segment of finite length and an infinite ray. Sketch this net in a labelled diagram.

	\hfill{(2000 - 10 Marks)}

\item Let $ABC$ and $PQR$ be any two triangles in the same plane.
	Assume that the perpendiculars from the points $\vec{A, B, C}$ to 
		the sides $QR, RP, PQ$ respectively are concurrent. Using vector methods or otherwise, prove that the perpendiculars from $\vec{P, Q, R}$ to $BC, CA, AB$ respectively are also concurrent. 

	\hfill{(2000 - 10 Marks)}

\item Let $a, b, c$ be real numbers with $a^{2}+b^{2}+c^{2}=1$. Show that the equation 
	\begin{align}
		\mydet{
ax-by-c     &bx+ay       &cx+a      \\
bx+ay       &-ax+by-c   &cy+b       \\
cx+a         &cy+b         &-ax-by+c \\} =0
	\end{align}
represents a straight line.

\hfill{(2001 - 6 Marks)}

\item A straight line $L$, through the origin meets the lines $x+y=1$ and $x+y=3$ at $\vec{P}$ and $\vec{Q}$ respectively. Through $\vec{P}$ and $\vec{Q}$ two 
	straight lines $L_{1}$, and $L_{2}$ are drawn, parallel to $2x-y=5$ and $3x+y=5$ respectively. Lines $L_{1}$ and $L_{2}$ intersect at $\vec{R}$. Show 
		that the locus of $\vec{R}$, as $L$, varies, is a straight line. 
\hfill{(2002 - 5 Marks)}

\item A straight line $L$ with negative slope passes through the 
point \brak{8, 2} and cuts the positive coordinate axes at points 
		$\vec{P}$ and $\vec{Q}$. Find the absolute minimum value of $OP + OQ$, as $L$ 
varies, where $O$ is the origin.

\hfill{(2002 - 5 Marks)}

\item The area of the triangle formed by the intersection of a line 
	parallel to x-axis and passing through $\vec{P}\brak{h, k}$ with the lines 
$y=x$ and $x+y=2$ is $4h^{2}$, Find the locus of the point P.

\hfill{(2005 - 2 Marks)}

%\end{enumerate}


   
