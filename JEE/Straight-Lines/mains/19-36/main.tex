\iffalse
\title{Assignment-3 }
\author{AI24BTECH11036- Shreedhanvi Yadlapally}
\section{mains}
\fi
% \begin{enumerate}
	\item Let $\vec{P}=\brak{-1, 0}$, $\vec{Q}=\brak{0, 0}$ and $\vec{R}=\brak{3, 3\sqrt{3}}$ be three points. The equation of the bisector of the angle $PQR$ is \hfill{[2007]}
\begin{multicols}{2}
\begin{enumerate}
\item $\frac{\sqrt{3}}{2}x+y=0$
\item $x+\sqrt{3}y=0$
\item $\sqrt{3}x+y=0$
\item $x+\frac{\sqrt{3}}{2}y=0$
\end{enumerate}
\end{multicols}

\item If one of the lines of $my^{2}+(1-m^{2})xy-mx^{2}=80$ is a bisector 
of the angle between the lines $xy=0$, then m is

\hfill{[2007]}
\begin{multicols}{4}
\begin{enumerate}
\item 1
\item 2
\item -$\frac{1}{2}$
\item -2
\end{enumerate}
\end{multicols}

\item The perpendicular bisector of the line segment joining $\vec{P}\brak{1,4}$ and $\vec{Q}\brak{k, 3}$ has y-intercept -4. Then a possible value of $k$ is \hfill{[2008]}
\begin{multicols}{4}
\begin{enumerate}
\item 1
\item 2
\item -2
\item -4
\end{enumerate}
\end{multicols} 

\item The shortest distance between the line $y- x =1$ and the 
curve $x=y^{2}$ is \hfill{[2009]}
\begin{multicols}{4}
\begin{enumerate}
\item $\frac{2\sqrt{3}}{8}$
\item $\frac{3\sqrt{2}}{5}$
\item $\frac{\sqrt{3}}{4}$
\item $\frac{3\sqrt{2}}{8}$
\end{enumerate}
\end{multicols}

\item The lines $p\brak{p^{2}+1}x-y+q=0$ and $\brak{p^{2}+1}^{2}x+\brak{p^{2}+1}y+ 2q=0$ are perpendicular to a common line for : \hfill{[2009]}
\begin{enumerate}
\item exactly one values of $p$
\item exactly two values of $p$ 
\item more than two values of $p$ 
\item no value of $p$ 
\end{enumerate}

\item Three distinct points $\vec{A}$, $\vec{B}$ and $\vec{C}$ are given in the 
2-dimensional coordinates plane such that the ratio of the 
distance of any one of them from the point $\brak{1, 0}$ to the distance from
the point $\brak{-1, 0}$ is equal to $\frac{1}{3}$. Then the circumcentre of the triangle $ABC$ is at the point: \hfill{[2009]}
\begin{multicols}{4}
\begin{enumerate}
\item $\brak{\frac{5}{4}, 0}$
\item $\brak{\frac{5}{2}, 0}$
\item $\brak{\frac{5}{3}, 0}$
\item $\brak{0, 0}$
\end{enumerate}
\end{multicols} 

\item The line $L$ given by $\frac{x}{5}+\frac{y}{b}=1$ passes through the point $\brak{13, 32}$. The line $K$ is parallel to the line $L$ and has the equation $\frac{x}{c}+\frac{y}{3}=1$. Then the distance between $L$ and $K$ is
\hfill{[2010]}
\begin{multicols}{4}
\begin{enumerate}
\item $\sqrt{17}$
\item $\frac{17}{\sqrt{15}}$
\item $\frac{23}{\sqrt{17}}$
\item $\frac{23}{\sqrt{15}}$
\end{enumerate}
\end{multicols} 

\item Lines $L_{1}: y-x=0$ and $L_{2}: 2x+y=0$ intersect the line $L_{3}: y+2=0$ at $\vec{P}$ and $\vec{Q}$, respectively. The bisector of the acute 
	angle between $L_{1}$ and $L_{2}$ intersects $L_{3}$ at $\vec{R}$.\\
\textbf{STATEMENT-1 :} The ratio $PR:RQ$ equals $2\sqrt{2}:\sqrt{5}$.\\
\textbf{STATEMENT-2 :} In any triangle, bisector of an angle divides the triangle into two similar triangles.

\hfill{[2011]}
   \begin{enumerate}
   \item Statement-1 is True, Statement-2 is True Statement-2 
is not a correct explanation for Statement-1 
   \item Statement-1 is True, Statement-2 is True; Statement-2 
is NOT a correct explanation for Statement-1 
   \item Statement-I is True, Statement-2 is False
   \item Statement-1 is False, Statement-2 is True. 
   \end{enumerate}

\item If the line $2x+y=k$ passes through the point which divides the line segment joining the points $\brak{1, 1}$ and $\brak{2, 4}$ in the ration 3:2, then $k$ equals: \hfill{[2012]}
\begin{multicols}{4}
\begin{enumerate}
\item $\frac{29}{5}$
\item 5
\item 6
\item $\frac{11}{5}$
\end{enumerate}
\end{multicols} 

\item A ray of light along $x+\sqrt{3}y=\sqrt{3}$ gets reflected upon reaching the x-axis, the equation of the reflected ray is \hfill{[JEE M 2013]}
\begin{multicols}{2}
\begin{enumerate}
\item $y=x+\sqrt{3}$
\item $\sqrt{3}y=x-\sqrt{3}$
\item $y=\sqrt{3}x-\sqrt{3}$
\item $\sqrt{3}y=x-1$
\end{enumerate}
\end{multicols}

\item The x-coordinate of the incentre of the triangle that has the coordinates of mid points of its sides as $\brak{0, 1}$, $\brak{1, 1}$ and $\brak{1, 0}$ is:

\hfill{[JEE M 2013]}
\begin{multicols}{4}
\begin{enumerate}
\item $2+\sqrt{2}$
\item $2-\sqrt{2}$
\item $1+\sqrt{2}$
\item $1-\sqrt{2}$
\end{enumerate}
\end{multicols}

\item Let $PS$ be the median of the triangle with vertices $\vec{P}\brak{2, 2}$, $\vec{Q}\brak{6, -1}$ and $\vec{R}\brak{7,3}$. The equation of the line passing through $\brak{1, -1}$ and parallel to $PS$ is: \hfill{[JEE M 2014]}
\begin{multicols}{2}
\begin{enumerate}
\item $4x+7y+3=0$
\item $2x-9y-11=0$
\item $4x-7y-11=0$
\item $2x+9y+7=0$
\end{enumerate}
\end{multicols}

\item Let $a, b, c$ and $d$ be non-zero numbers. If the point of intersection of the lines $4ax+2ay+c=0$ and $5bx+2by+d=0$ lies in the fourth quadrant and is equidistant from the two axes then \hfill{[JEE M 2014]}
\begin{multicols}{2}
\begin{enumerate}
\item $3bc-2ad= 0$
\item $3bc+2ad=0$
\item $2bc-3ad= 0$
\item $2bc+3ad=0$ 
\end{enumerate}
\end{multicols}

\item The number of points, having both co-ordinates as integers, 
that lie in the interior of the triangle with vertices $\brak{0, 0}$,  $\brak{0,41}$ and $\brak{41, 0}$ is: 

\hfill{[JEE M 2015]}
\begin{multicols}{4}
\begin{enumerate}
\item 820
\item 780
\item 901
\item 861
\end{enumerate}
\end{multicols}

\item Two sides of a rhombus are along the lines, $x-y+1=0$ and 
$7x-y-5=0$. If its diagonals intersect at $\brak{-1, -2}$, then which one of the following is a vertex of this rhombus?

\hfill{[JEE M 2016]}
\begin{multicols}{2}
\begin{enumerate}
\item $\brak{\frac{1}{3}, -\frac{8}{3}}$
\item $\brak{-\frac{10}{3}, -\frac{7}{3}}$
\item $\brak{-3, -9}$
\item $\brak{-3, -8}$
\end{enumerate}
\end{multicols}

\item A straight the through a fixed point $\brak{2, 3}$ intersects the 
	coordinate axes at distinct points $\vec{P}$ and $\vec{Q}$. If $\vec{O}$ is the origin 
and the rectangle $OPRQ$ is completed, then the locus of $R$ is: 

\hfill{[JEE M 2018]}
\begin{multicols}{2}
\begin{enumerate}
\item $2x+3y = xy$
\item $3x+2y = xy$ 
\item $3x+2y = 6xy$ 
\item $3x+2y = 6$
\end{enumerate}
\end{multicols}

\item Consider the set of all lines $px+qy+r=0$ such that 
$3p+2q+4r=0$. Which one of the following statements is true? 

\hfill{[JEE M 2019- 9 Jan (M)]}
\begin{enumerate}
\item The lines are concurrent at the point $\brak{\frac{3}{4}, \frac{1}{2}}$
\item Each line passes through the origin. 
\item The lines are all parallel.
\item The lines are not concurrent.
\end{enumerate}

\item Slope of a line passing through $\vec{P}\brak{2, 3}$ and intersecting the line $x+y=7$ at a distance of 4 units from $\vec{P}$, is:

\hfill{[JEE M 2019- 9 April(M)]}
\begin{multicols}{2}
\begin{enumerate}
\item $\frac{1-\sqrt{5}}{1+\sqrt{5}}$
\item $\frac{1-\sqrt{7}}{1+\sqrt{7}}$
\item $\frac{\sqrt{7}-1}{\sqrt{7}+1}$
\item $\frac{\sqrt{5}-1}{\sqrt{5}+1}$
\end{enumerate}
\end{multicols}

% \end{enumerate}
% \end{document} 

