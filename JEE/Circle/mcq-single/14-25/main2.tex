\iffalse
\title{8.CIRCLE}
\author{EE24BTECH11023-RASAGNA}
\section{mcq-single}
\fi
    %\begin{enumerate}
        \item The triangle $PQR$ is inscribed in the circle $x^2+y^2=25$.If $Q$ and $R$ have co-ordinates $\brak{3,4}$ and $\brak{-4,3}$ respectively,then $\angle$QPR is equal to  
    \hfill$\brak{2000S}$
    
    
    \begin{multicols}{2}
    \begin{enumerate}
    
        \item $\frac{\pi}{2}$
        \item $\frac{\pi}{3}$
        \item $\frac{\pi}{4}$
        \item $\frac{\pi}{6}$
    \end{enumerate}
    \end{multicols}
     \item If the circles $x^2+y^2+2x+2ky+k=0$ intersect orthogonally,then $k$ is
        \hfill$\brak{2000S}$
    \begin{multicols}{2}
    \begin{enumerate}
        \item 2 or $-\frac{3}{2}$
        \item -2 or $-\frac{3}{2}$
        \item 2 or $\frac{3}{2}$
        \item $-$2 or $\frac{3}{2}$
    \end{enumerate}
    \end{multicols}
    \item Let $AB$ be a chord of the circle $x^2+y^2=r^2$ subtending a right angle at the centre.then the locus of the centroid of the triangle $PAB$  as $P$ moves on the circle is 
        \hfill$\brak{2001S}$
        \begin{multicols}{2}
    \begin{enumerate}
        \item a parabola
        \item a circle
        \item an ellipse
        \item a pair of straight lines
        \end{enumerate}
        \end{multicols}
        \item Let $PQ$ and $RS$ be tangents at the extremities of the diameter PR of a circle of radius $r$. If $PS$ and $RQ$ intersect at a point $X$ on the circumference of the circle, then $2r$ equals
        \hfill$\brak{2001S}$
        \begin{multicols}{2}
        \begin{enumerate}
    \item $\sqrt{PQ.RS}$
     \item $\brak{PQ+RS}$
    \item $2PQ.RS/(PQ+RS)$
     \item {$\sqrt{(PQ^2+RS^2)}$}/2
     \end{enumerate}
     \end{multicols}
     \item If the tangent at the point $P$ on the circle $x^2+y^2+6x+6y=2$ meets a straight line $5x-2y+6=0$ at a point on the y-axis, then the length of $PQ$ is 
             \hfill$\brak{2002S}$
             \begin{multicols}{2}
         \begin{enumerate}
             \item 4
             \item 2$\sqrt5$
             \item 5
             \item 3$\sqrt5$
\end{enumerate}
\end{multicols}
     \item The centre of the circle inscribed in square formed by the lines $x^2-8x+12=0$ and $y^2-14y+45=0$, is
         \hfill$\brak{2003S}$
         \begin{multicols}{2}
     \begin{enumerate}
         \item $\brak{4,7}$
         \item $\brak{7,4}$
         \item $\brak{9,4}$
         \item $\brak{4,9}$
     \end{enumerate}
     \end{multicols}
     \item If one of the diameters of the circle $x^2+y^2-2x-6y+6=0$ is a chord to the circle with the centre $\brak{2,1}$,then the radius of the circle is 
         \hfill$\brak{2004S}$
         \begin{multicols}{2}
     \begin{enumerate}
         \item $\sqrt3$
         \item $\sqrt2$
         \item 3
         \item 2
     \end{enumerate}
     \end{multicols}
     \item A circle is given by $x^2+$\brak{y-1}$^2=1$, another circle $C$ touches it externally and also the x-axis, then the locus of its centre is
         \hfill$\brak{2005S}$
         \begin{multicols}{2}
     \begin{enumerate}
         \item \{$\brak{x,y}$:$x^2=4y$\} $\bigcup$ \{$\brak{x,y}$:y$\le$0\}
         \item \{$\brak{x,y}$:$x^2+(y-1)^2=4$\} $\bigcup$ \{$\brak{x,y}$:y$\le$0\}
         \item \{$\brak{x,y}$:$x^2=y$\} $\bigcup$ \{$\brak{0,y}$:y$\le$0\}
         \item \{$\brak{x,y}$:$x^2=4$y\} $\bigcup$ \{$\brak{0,y}$:y$\le$0\}
         \end{enumerate}
         \end{multicols}
         \item Tangents drawn from the point $P\brak{1,8}$ to the circle $x^2+y^2-6x-4y-11=0$ touch the circle at the points $A$ and $B$. The equation of the circumcircle of the triangle $PAB$ is
             \hfill$\brak{2009}$
             \begin{multicols}{2}
         \begin{enumerate}
             \item $x^2+y^2+4x-6y+19=0$
             \item $x^2+y^2-4x-10y+19=0$
             \item $x^2+y^2-4x+6y-29=0$
             \item $x^2+y^2-4x-6y+19=0$
             \end{enumerate}
             \end{multicols}
             \item The circle passing through the point $\brak{-1,0}$ and touching the y-axis at $\brak{0,2}$ also passes through the point
                 \hfill$\brak{2011}$
                 \begin{multicols}{2}
             \begin{enumerate}
                 \item $\brak{-\frac{3}{2},0}$
                 \item $\brak{-\frac{5}{2},2}$
                 \item $\brak{-\frac{3}{2},\frac{5}{2}}$
                 \item $\brak{-4,0}$
             \end{enumerate}
             \end{multicols}
             \item The locus of the mid-point of the chord of contact of tangents drawn from points lying on the straight line $4x-5y=20$ to the circle $x^2+y^2=9$ is
                 \hfill$\brak{2012}$
                 \begin{multicols}{2}
             \begin{enumerate}
                 \item $20\brak{x^2+y^2}-36x+45y=0$
                 \item $20\brak{x^2+y^2}+36x-45y=0$
                 \item $36\brak{x^2+y^2}-20x+45y=0$
                 \item $36\brak{x^2+y^2}+20x-45y=0$
             \end{enumerate}
             \end{multicols}
             \item A line $y=mx+1$ intersects the circle$(x-3)^2+(y+2)^2=25$ at the points $P$ and $Q$. if the mid point of the line segment $PQ$ has x-coordinate $-\frac{3}{5}$, then which one of the following options is correct?
                 \hfill$\brak{JEE Adv. 2019}$
                 \begin{multicols}{2}
             \begin{enumerate}
                 \item $2\le m<4$
                 \item $-3\le m<-1$
                 \item $4\le m<6$
                 \item $6\le m<8$
             \end{enumerate}
             \end{multicols}
             

     
     

%\end{enumerate}

        
